
\documentclass[12pt,a4paper,euler,beton]{ata} 
\usepackage[brazil]{babel}
\usepackage[utf8]{inputenc} 
\usepackage[T1]{fontenc}
\usepackage{url}

\begin{document}


% Data da reunião.
\dia{26}
\mes{11}
\ano{2012}


% Tipo da reunião.
\deftiporeuniao{Ordinária}
% Advérbio correspondente ao tipo da reunião.
\deftiporeuniaoadv{ordinariamente}


% Cabeçalho com quebras de linhas manuais. 
% Basicamente sempre o mesmo.
% O comando \hoje gera a data (em numerais). 
\cabecalho{%
  Ata da Reunião \tiporeuniao{} do \\ 
  Departamento de Física e Matemática (RFM),\\
  do Instituto de Ciência e Tecnologia do PURO,\\
  Universidade Federal Fluminense\\
  realizada em \hoje%
}


% Quem presidiu a reunião (sem título).
\defpresidiu{Fernando Náufel do Amaral}


% Cargo de quem presidiu a reunião.
\defcargopresidiu{terceiro guru de \LaTeX{} do RFM}


% Quem secretariou (sem título).
\defsecretariou{Fernando Náufel do Amaral}


% Lista de presentes. NÃO terminar com ponto.
\defpresentes{
  Donald Knuth,

  Leslie Lamport,

  Hermann Zapf,

  e Dave Bowman
}


% Lista de ausentes justificados. NÃO terminar com ponto.
\defjustif{
  Richard Feynman,

  Alan Turing,

  Isaac Newton,

  e HAL~$9001$
}


% Lista de ausentes. NÃO terminar com ponto.
\defausentes{
  José Sarney,

  e Immanuel Velikovsky
}


% Detalhamento da pauta. Cada ponto é um ambiente da forma
%
%    \begin{ponto}{Nome do ponto}
%    
%      Corpo do ponto.
%    
%    \end{ponto}
%
% O nome do ponto não deve terminar com pontuação. O corpo sim.
% Quebras de linha e listas de itens podem ser inseridas à vontade,
% mas a ata inteira será diagramada como um único parágrafo.

\begin{ponto}{Objetivos da classe \texttt{ata}}

  Depois de escrever algumas atas de reuniões departamentais, o~\prof
  Náufel constatou o grande sofrimento que tal tarefa pode causar.
  Os~detalhes de formatação são tediosos, assim como é a repetição de
  trechos (e.g., os nomes dos pontos na apresentação da pauta e no
  desenvolvimento da reunião).

  Como programar em \LaTeX{} pode ser divertido, o \prof Náufel
  decidiu escrever a classe \texttt{ata}. (Programar em \LaTeX{}
  também pode ser enervante às vezes, mas nem de longe tão enervante
  quanto escrever atas.) O~\prof Náufel espera que esta classe
  facilite a confecção e a padronização das atas das nossas reuniões
  departamentais, e que, ao liberar o autor da ata das tarefas mais
  chatas da \emph{formatação} das atas, esta classe contribua para a
  boa qualidade do seu \emph{conteúdo}.

  Este documento explica e ilustra o uso da classe \texttt{ata}.
  Examine o arquivo fonte \texttt{exemploata.tex} para ver exemplos do
  uso dos recursos da classe \texttt{ata}.

\end{ponto}


\begin{ponto}[inst]{Instalação da classe \texttt{ata}}
  
  \begin{enumerate}

  \item Para usar a classe \texttt{ata}, basta copiar o arquivo
    \texttt{ata.cls} para algum diretório onde sua instalação \LaTeX{}
    o encontre. Isto depende do sistema operacional e da distribuição
    de \LaTeX{} que você estiver usando. Na WWW, em
    \url{en.wikibooks.org/wiki/LaTeX/Packages/Installing_Extra_Packages},\\
    por exemplo, você pode encontrar instruções detalhadas. Observe
    que o~arquivo \texttt{ata.cls} está sendo disponibilizado já
    pronto para uso; não é preciso extrair nada, nem criar a
    documentação.

  \item A classe \texttt{ata} depende dos pacotes \texttt{ifthenelse}
    (para testar condições), \texttt{paralist} (para gerar listas
    numeradas e não numeradas sem quebra de parágrafos), e
    \texttt{lineno} (para numerar as linhas do corpo da ata). Visite
    \url{http://www.ctan.org/} para obter estes pacotes, caso sua
    instalação \LaTeX{} não os inclua.

  \item Caso você deseje usar os fontes \emph{Concrete} e \emph{Euler}
    (veja o ponto~\ref{opcoes}, item~\ref{fonts}, abaixo), os
    respectivos pacotes (\texttt{beton} e \texttt{euler}) também devem
    estar instalados.

  \end{enumerate}
  
\end{ponto}


\begin{ponto}[opcoes]{Invocação da classe ata}

  \begin{enumerate}

  \item A classe \texttt{ata} aceita todas as opções da classe
    \texttt{article}.

  \item \label{fonts} Além delas, podem ser passadas as seguintes
    opções:

    \begin{enumerate}[(i)]

    \item \texttt{euler}: faz com que seja usado os fontes matemáticos
      Euler, criados pelo \emph{designer} de fontes Hermann Zapf em
      colaboração com Donald Knuth (autor de \TeX). O~fonte Euler foi
      usado para produzir, entre outros, o livro \emph{Concrete
        Mathematics} (Addison-Wesley, $1989$), de autoria de Graham,
      Knuth, e Patashnik. Os capítulos~$17$ e~$18$ do livro
      \emph{Digital Typography} (CSLI, $1999$), de Donald E.~Knuth,
      relatam a história da criação dos fontes. Embora a necessidade
      de fontes matemáticos no texto de uma ata seja altamente
      questionável, apenas a qualidade estética superior dos
      algarismos já justifica esta opção. Compare você mesmo os
      dígitos no fonte usual do \LaTeX{} (1234567890) com os dígitos
      em Euler ($1234567890$). No~entanto, é preciso atentar para um
      detalhe importante: \emph{só são diagramados em Euler os
        algarismos que estiverem em modo matemático}. Para garantir
      que os números apareçam no fonte Euler, coloque-os entre um par
      de~\$.

    \item \texttt{beton}: como o fonte Euler é um pouco mais pesado do
      que fonte matemático normal, Knuth achou necessário desenvolver
      um fonte de texto compatível, também mais escuro, chamado
      \emph{Concrete}. A~classe \texttt{ata} comporta a opção
      \texttt{beton}, que faz com que esse fonte seja usado em todo o
      texto da ata. Para um resultado melhor, você deve usar ou as
      duas opções (\texttt{euler} e \texttt{beton}), ou nenhuma.

    \end{enumerate}
    
  \end{enumerate}

\end{ponto}


\begin{ponto}{Comandos para processar a data}

  Os~comandos \verb+\dia+, \verb+\mes+, e \verb+\ano+, todos de um
  argumento numérico obrigatório, fazem o óbvio: definem dia, mês e
  ano. Você deve passar os argumentos na forma de valores numéricos
  (i.e., \emph{não} em modo matemático). Após definir dia, mês e ano,
  você pode usar, no cabeçalho e no corpo da ata, o comando
  \verb+\hoje+ para produzir a data no formato \texttt{dd/mm/aaaa} e o
  comando \verb+\hojeporextenso+ para produzir a data no formato
  \texttt{Aos dd de mmmmmmm de aaaa} (com o nome do mês por extenso).

\end{ponto}


\begin{ponto}{Comandos relativos ao tipo de reunião}

  O~comando \verb+\deftiporeuniao+, de um argumento obrigatório,
  define o tipo da reunião (ordinária ou extraordinária). O comando
  \verb+\deftiporeuniaoadv+, também de um argumento obrigatório,
  define o advérbio correspondente (ordinariamente ou
  extraordinariamente). No corpo da ata, essa informação pode ser
  produzida com os comandos \verb+\tiporeuniao+ e
  \verb+\tiporeuniaoadv+, respectivamente.

\end{ponto}


\begin{ponto}{Comandos para gerar o cabeçalho}

  O~comando \verb+\cabecalho+, de um argumento obrigatório, define o
  cabeçalho da ata (que é diagramado de forma análoga ao título pela
  classe \texttt{article}). Caso você deseje quebras de linha em
  locais específicos do cabeçalho, você deve inseri-las manualmente no
  argumento deste comando.

\end{ponto}


\begin{ponto}{Comandos relativos à presidência e à secretaria}

  Os~comandos \verb+\defpresidiu+, \verb+\defcargopresidiu+, e
  \verb+\defsecretariou+, todos de um argumento obrigatório, definem
  os nomes das pessoas que desempenharam essas funções na reunião, bem
  como o cargo de quem presidiu. No corpo da ata, esses nomes podem
  ser produzidos com os comandos \verb+\presidiu+,
  \verb+\cargopresidiu+, e \verb+\secretariou+, respectivamente.

\end{ponto}


\begin{ponto}{Listas de presentes e ausentes}

  Os~comandos \verb+\defpresentes+, \verb+\defausentes+, e
  \verb+\defjustif+, todos de um argumento obrigatório, definem as
  listas das pessoas que estavam presentes, ausentes, e ausentes com
  justificativa, respectivamente. No corpo da ata, essas listas podem
  ser produzidas com os comandos \verb+\presentes+, \verb+\ausentes+,
  e \verb+\justificados+, respectivamente. Ao~definir essas listas,
  você pode usar quebras de linha à vontade, mas \emph{não termine a
    lista com ponto final}.

\end{ponto}


\begin{ponto}{Comandos e ambiente para pontos da pauta}

  Este é o recurso principal da classe \texttt{ata}. Cada ponto de
  pauta é definido através do ambiente \texttt{ponto}, de um argumento
  obrigatório, da seguinte forma:
  \verb+\begin{ponto}{xxx} yyy \end{ponto}+, onde \verb+xxx+ é o nome
  do ponto, e \verb+yyy+ o desenvolvimento do ponto. A numeração dos
  pontos é automática. Observe que o nome do ponto é um argumento
  obrigatório. O nome do ponto deve começar por maiúscula. Não termine
  o nome do ponto com qualquer pontuação. O desenvolvimento do ponto pode
  incluir quebras de linhas, itemizações, enumerações, etc. (veja o
  ponto~\ref{listas} abaixo). Caso você deseje definir um \emph{label}
  para que o ponto possa ser referenciado pelo seu número a partir de
  outros locais do texto, passe esse \emph{label} para o argumento
  \texttt{ponto} na forma de um argumento opcional, antes do título:
  \verb+\begin{ponto}[lll]{xxx} yyy \end{ponto}+. No corpo da ata, o
  comando \verb+\pauta+ produz a lista numerada dos nomes dos pontos,
  cada nome seguido de vírgula (com exceção do último, que é seguido
  de ponto final). O~comando \verb+\desenvolvimento+ produz a lista
  numerada dos nomes dos pontos da pauta, seguidos de dois pontos (:),
  seguidos dos desenvolvimentos dos pontos da pauta.

\end{ponto}


\begin{ponto}[listas]{Enumerações, listas de itens e descrições} 

  Os~ambientes \texttt{itemize}, \texttt{enumerate} e
  \texttt{description} podem ser usados normalmente dentro do
  desenvolvimento de um ponto. Estes ambientes foram redefinidos para
  não produzir quebras de linhas, o que não impede você de diagramar o
  seu fonte \LaTeX{} do jeito que você achar mais conveniente, com
  linhas em branco entre itens, etc. Por \emph{default}, os pontos da
  pauta são rotulados com números arábicos, e os itens de listas
  enumeradas dentro de um ponto são rotulados com letras minúsculas.
  Caso você deseje usar outro tipo de rótulo para os itens de uma
  enumeração, leia a documentação do pacote \texttt{paralist}.
  Basicamente, você deve passar um argumento opcional para o ambiente
  \verb+enumerate+. Por exemplo, a lista enumerada no item~\ref{fonts}
  do ponto~\ref{opcoes} desta ata foi rotulada com números romanos
  minúsculos através do seguinte código:
  \verb+\begin{enumerate}[(i)] ... \end{enumerate}+. Caso você deseje
  referenciar um item de uma lista enumerada, use o~comando
  \verb+\label+ normalmente.

\end{ponto}


\begin{ponto}{Abreviaturas para ``Professor'' e ``Professora''}

  Para garantir o espacejamento correto (``\prof Náufel'' em oposição
  a ``Prof. Náufel''), use o~comando \verb+\prof+ para produzir a
  abreviatura \prof e o~comando \verb+\profa+ para produzir a
  abreviatura \profa. Deixe apenas espaços entre o~comando e o~nome
  do(a) professor(a).

\end{ponto}


\begin{ponto}{Configurações}

  Por enquanto, a única opção de configuração é o~comando
  \verb+\estilonomeponto+, definido por \emph{default} como
  \verb+\bfseries+, que determina o estilo usado para produzir os
  nomes dos pontos, tanto no comando \verb+\pauta+ quanto no comando
  \verb+\desenvolvimento+. Para alterar esse estilo, use
  \verb+\renewcommand{\estilonomeponto}{...}+.

\end{ponto}


\begin{ponto}{Licença}

  A classe \texttt{ata} está sendo disponibilizada segundo a licença
  \emph{LaTeX Project Public License}, que pode ser obtida em
  \url{http://www.latex-project.org/lppl.txt}, na sua versão $1.3$ ou
  mais recente.

\end{ponto}


\begin{ponto}{Contato, dúvidas, relatos de \emph{bugs} e sugestões}

  Por favor, envie e-mail para \url{fnaufel@gmail.com}.

\end{ponto}



%%%%%%%%%%%%%%%%%%%%%%%%%%%%%%%%%%%%%%%%%%%%%%%%%%%%%%%%%%%%%%%%
%
% Corpo da ata. Basicamente sempre o mesmo. Verificar se a pauta foi
% aprovada por unanimidade. Se não houver faltantes, eliminar a(s)
% linha(s) correspondente(s).

\hojeporextenso, reuniram-se \tiporeuniaoadv{} os professores do
Departamento de Física e Matemática (RFM) do Instituto de Ciência e
Tecnologia do PURO/Universidade Federal Fluminense.
%
Compareceram os professores \presentes. 
%
Justificaram ausência os professores \justificados.
%
Faltaram os professores \ausentes.
%
Abrindo a reunião, o \prof \presidiu, \cargopresidiu, apresentou a
seguinte proposta de pauta, que foi aprovada por unanimidade pela
plenária:
%
\pauta
%
A reunião se desenvolveu conforme se segue:
% 
\desenvolvimento
% 
Não havendo mais a tratar, a reunião foi encerrada, da qual eu,
\secretariou, lavrei a presente ata, que vai assinada por mim e pelo
chefe do departamento.
%
\end{document}
